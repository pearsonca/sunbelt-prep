%%This is a very basic article template.
%%There is just one section and two subsections.
\documentclass{article}

% PLOS One TEMPLATE
\usepackage{amsmath,amssymb,graphicx,cite,setspace}
\doublespacing

% Text layout
\topmargin 0.0cm
\oddsidemargin 0.5cm
\evensidemargin 0.5cm
\textwidth 16cm 
\textheight 21cm

\usepackage[labelfont=bf,labelsep=period,justification=raggedright]{caption}
%%%%%%%

\begin{document}
\begin{flushleft}
{\Large
\textbf{Network Class Superposition Analyses}
}
% Insert Author names, affiliations and corresponding author email.
\\
Carl A. B. Pearson$^{1,2,\ast}$, 
Edo Airoldi$^{2}$, 
Edward Kao$^{2}$,
Burton Singer$^{1}$, 
\\
\bf{1} Emerging Pathogens Institute, University of Florida, Gainesville, FL, USA
\\
\bf{2} Statistics, Harvard University, Cambridge, MA, USA
\\
$\ast$ E-mail: cap10@ufl.edu
\end{flushleft}
% Please keep the abstract between 250 and 300 words
\section*{Abstract}
We specify and demonstrate a graph-based model of populace-wide communications, with an embedded, relatively small module representing a clandestine group.  The members of this group behave similarly to background population, except they also pass messages in furtherance of some plot.  We parametrize this model based on cell phone data sets.

The purpose of this model is to provide a test framework for various methods of detecting these clandestine groups, in real time, before they achieve the intended plot.  We refer to collection of such methods as an {\em Observer}.  We propose various Observer models and measure their performance relative to statistical features of the population, plotters, and their respective communication behaviors.

Finally, we consider the implications of {\em decoy} messages.  In the basic model, we consider missing -- but not misleading or false -- communications.  If, instead, the plotters or the Observer can issue forged messages, the problem becomes substantially more complex.
\section{Title}

\subsection{Subtitle}

Plain text.

\subsection{Another subtitle}

More plain text.


\end{document}
